%%%%%%%%%%%%%%%%%%%%%%%%%%%%%%%%%%%%%%%%%%%%%%%%%%%%%%%%%%%%%%%%%%%%%%
% Source: Dave Richeson (divisbyzero.com), Dickinson College
% Version francaise par Vincent Pantaloni, prof.pantaloni.free.fr
% Traduction, correction et adaptation à la typographie française.
% 
% Une anti-seche en deux pages pour une intro rapide ou un aide mémoire des différentes fonctions. A imprimer en recto verso par exemple.
%
% Feel free to distribute this example, but please keep the referral
% to divisbyzero.com
% 
%%%%%%%%%%%%%%%%%%%%%%%%%%%%%%%%%%%%%%%%%%%%%%%%%%%%%%%%%%%%%%%%%%%%%%
%
%%%%%%%%%%%%%%%%%%%%%%%%%%%%%%%%%%%%%%%%%%%%%%%%%%%%%%%%%%%%%%%%%%%%%%

\documentclass[a4paper,10pt,landscape]{article}
\usepackage{fontspec}
\usepackage{amssymb,amsmath,amsthm,amsfonts}
\usepackage{multicol,multirow}
\usepackage{calc}
\usepackage{ifthen}
\usepackage{microtype}
\usepackage{url}
\usepackage[landscape]{geometry}
\usepackage[colorlinks=true,citecolor=blue,linkcolor=blue]{hyperref}
\usepackage{tikz}
\usepackage[most]{tcolorbox}
\usepackage{xcolor}

% Define custom colors
\definecolor{warningyellow}{RGB}{255,204,0}
\definecolor{warningred}{RGB}{204,0,0}

% Create a new tcolorbox style for warnings
\newtcolorbox{warningbox}[1][]{%
  enhanced,
  colback=warningyellow!10,
  colframe=warningred,
  boxrule=1pt,
  title=Warning,
  fonttitle=\bfseries,
  coltitle=white,
  top=2mm,
  bottom=2mm,
  left=2mm,
  right=2mm,
  arc=1mm,
  title style={top color=warningred,bottom color=warningred},
  #1
}
\usetikzlibrary{arrows.meta}

\ifthenelse{\lengthtest { \paperwidth = 11in}}
    { \geometry{top=.5in,left=.5in,right=.5in,bottom=.5in} }
    {\ifthenelse{ \lengthtest{ \paperwidth = 297mm}}
        {\geometry{top=1cm,left=1cm,right=1cm,bottom=1cm} }
        {\geometry{top=1cm,left=1cm,right=1cm,bottom=1cm} }
    }
\pagestyle{empty}
\makeatletter
\renewcommand{\section}{\@startsection{section}{1}{0mm}%
                                {-1ex plus -.5ex minus -.2ex}%
                                {0.5ex plus .2ex}%x
                                {\normalfont\large\bfseries}}
\renewcommand{\subsection}{\@startsection{subsection}{2}{0mm}%
                                {-1explus -.5ex minus -.2ex}%
                                {0.5ex plus .2ex}%
                                {\normalfont\normalsize\bfseries}}
\renewcommand{\subsubsection}{\@startsection{subsubsection}{3}{0mm}%
                                {-1ex plus -.5ex minus -.2ex}%
                                {1ex plus .2ex}%
                                {\normalfont\small\bfseries}}
\makeatother
\setcounter{secnumdepth}{0}
\setlength{\parindent}{0pt}
\setlength{\parskip}{0pt plus 0.5ex}

% -----------------------------------------------------------------------

\begin{document}

\raggedright
\footnotesize

\begin{center}
    {\Large\textbf{PC1101 Cheatsheet, by randomwish}}\\[0.5em]
    \url{https://github.com/randomwish/schoolNotes}
\end{center}
\begin{multicols}{3}
\setlength{\premulticols}{1pt}
\setlength{\postmulticols}{1pt}
\setlength{\multicolsep}{1pt}
\setlength{\columnsep}{2pt}

\section{1. Intro to light}
\subsection{Preliminaries}
\subsubsection{Electromagnetic field and Maxwell's equations}
Nomenclature
\begin{itemize}
    \item $\vec{E}$: direction along which a charged particle is pushed (polar vector). Perpendicular to direction of propagation.
    \item $\vec{D}$: electric displacement field.
    \item $\vec{H}$: magnetic field.
    \item $\vec{B}$: axis around which a charged particle will rotate (axial vector). Perpendicular to both $\vec{E}$ and direction of propagation.
    \item $\nabla$: derivatives with respect to spatial coordinates.
    \item $\rho$: charge density.
    \item $\vec{j}$: current density.
\end{itemize}

Equations
\begin{itemize}
    \item $\vec{\nabla} \times \vec{E}(\vec{x}, t) + \partial_{t} \vec{B}(\vec{x}, t) = 0$
    \begin{itemize}
        \item $\vec{E}$ circulates around any region where $\vec{B}$ is changing with time.
    \end{itemize}
    \item $\vec{\nabla} \cdot \vec{B}(\vec{x},t) = 0$
    \begin{itemize}
        \item No magnetic monopoles; net outflow of $\vec{B}$ is zero.
    \end{itemize}
    \item $\vec{\nabla} \cdot \vec{D}(\vec{x},t) = \rho(\vec{x},t)$
    \begin{itemize}
        \item The electric displacement field $\vec{D}$ relates to charge density.
    \end{itemize}
    \item $\vec{\nabla} \times \vec{H}(\vec{x},t) - \partial_{t} \vec{D}(\vec{x},t) = \vec{j}(\vec{x},t)$
    \begin{itemize}
        \item Describes the circulation of $\vec{H}$ around a current density $\vec{j}$ and the time derivative of $\vec{D}$.
    \end{itemize}
\end{itemize}

\subsection{E-m field with matter}
\subsubsection{Forces on charged particle}
In addition to factoring in Newton's equation with electric and Lorentz forces, include the Abraham-Lorentz force, which accounts for the energy radiated by an accelerated charge.

\subsubsection{Interaction with a molecule}
An electric dipole arises when positive and negative charges move in opposite directions, causing the centers of mass of both clouds to not overlap.

When an electric field oscillates, the dipole oscillates at the same frequency, emitting a wave at that frequency and scattering in all directions. Electrons are stable only in specific energy states. The energy difference between two states, $\Delta E_{jk} = E_j - E_k$, corresponds to a frequency via the relation
\[
\Delta E_{jk} = h\nu_{jk} = \hbar \omega_{jk},
\]
where $h$ is Planck's constant.

\subsubsection{Interaction with solids}
In dielectric materials, such as glass (where electrons are bound to their atoms), light scattered by each atom remains in phase with the light scattered by other atoms. As a result, light passing through the material continues to propagate in a coherent direction.

\section{Frequency}
The dispersion relation is given by $\lambda v = c$, where $c$ is the speed of light in vacuum ($3 \times 10^8$ m/s). Humans can perceive light only when its wavelength lies within the visible spectrum.

\section{2. Monochromatic Waves}
Monochromatic: single color $\rightarrow$ wave has a single frequency.
\begin{itemize}
    \item Frequency $v$: units in Hz.
    \item Phase $\varphi$: only matters if there are two or more waves (relative phase).
    \item Pulsation $\omega = 2 \pi v$: units in rad/s.
    \item Time dependence: $\cos(2\pi v t + \varphi) \equiv \cos(\omega t + \varphi)$.
    \item Conversion: $\sin(\theta) = \cos(\theta - \frac{\pi}{2})$.
\end{itemize}

\subsection{Space Dependence}
\subsubsection{1D Monochromatic Waves}
We represent space as $x - ct$, where $t$ represents time coordinates.

Space and time dependence of a 1D monochromatic wave:
\[
\cos(kx - \omega t + \varphi),
\]
where $k$ is the wave number:
\[
k = \frac{\omega}{c} = \frac{2\pi}{\lambda}.
\]
The minus sign indicates that the wave moves to the right (positive direction).

\subsubsection{3D Monochromatic Waves}
\begin{itemize}
    \item \textbf{Plane wave}: $\cos(\vec{k}\cdot\vec{x} - \omega t + \varphi)$.
    \begin{itemize}
        \item Wavefronts are planes; propagates eternally from $-\infty$ to $\infty$ in the direction $\hat{e_k}$.
        \item The space vibrates in unison in the perpendicular direction to $\hat{e_k}$.
    \end{itemize}
    \item \textbf{Spherical wave}: 
    \[
    \frac{1}{r(\vec{x})} \cos(kr(\vec{x}) - \omega t + \varphi),
    \]
    where $k$ is the wave number and $r(\vec{x})$ is the distance of $\vec{x}$ from the center $\vec{x_0}$ where the wave originates.
    \[
    r(\vec{x}) = \sqrt{(x - x_0)^2 + (y - y_0)^2 + (z - z_0)^2}
    \]
    \item wave fronts are surfaces of constant $r$, or spheres centered at $\vec{x_0}$
\end{itemize}

\end{multicols}
\end{document}
