%%%%%%%%%%%%%%%%%%%%%%%%%%%%%%%%%%%%%%%%%%%%%%%%%%%%%%%%%%%%%%%%%%%%%%
% Source: Dave Richeson (divisbyzero.com), Dickinson College
% Version francaise par Vincent Pantaloni, prof.pantaloni.free.fr
% Traduction, correction et adaptation à la typographie française.
% 
% Une anti-seche en deux pages pour une intro rapide ou un aide mémoire des différentes fonctions. A imprimer en recto verso par exemple.
%
% Feel free to distribute this example, but please keep the referral
% to divisbyzero.com
% 
%%%%%%%%%%%%%%%%%%%%%%%%%%%%%%%%%%%%%%%%%%%%%%%%%%%%%%%%%%%%%%%%%%%%%%
%
%%%%%%%%%%%%%%%%%%%%%%%%%%%%%%%%%%%%%%%%%%%%%%%%%%%%%%%%%%%%%%%%%%%%%%

\documentclass[a4paper,10pt,landscape]{article}
\usepackage{fontspec}
\usepackage{amssymb,amsmath,amsthm,amsfonts}
\usepackage{multicol,multirow}
\usepackage{calc}
\usepackage{ifthen}
\usepackage{microtype}
\usepackage{url}
\usepackage[landscape]{geometry}
\usepackage[colorlinks=true,citecolor=blue,linkcolor=blue]{hyperref}
\usepackage{tikz}
\usepackage[most]{tcolorbox}
\usepackage{xcolor}

% Define custom colors
\definecolor{warningyellow}{RGB}{255,204,0}
\definecolor{warningred}{RGB}{204,0,0}

% Create a new tcolorbox style for warnings
\newtcolorbox{warningbox}[1][]{%
  enhanced,
  colback=warningyellow!10,
  colframe=warningred,
  boxrule=1pt,
  title=Warning,
  fonttitle=\bfseries,
  coltitle=white,
  top=2mm,
  bottom=2mm,
  left=2mm,
  right=2mm,
  arc=1mm,
  title style={top color=warningred,bottom color=warningred},
  #1
}
\usetikzlibrary{arrows.meta}

\ifthenelse{\lengthtest { \paperwidth = 11in}}
    { \geometry{top=.5in,left=.5in,right=.5in,bottom=.5in} }
	{\ifthenelse{ \lengthtest{ \paperwidth = 297mm}}
		{\geometry{top=1cm,left=1cm,right=1cm,bottom=1cm} }
		{\geometry{top=1cm,left=1cm,right=1cm,bottom=1cm} }
	}
\pagestyle{empty}
\makeatletter
\renewcommand{\section}{\@startsection{section}{1}{0mm}%
                                {-1ex plus -.5ex minus -.2ex}%
                                {0.5ex plus .2ex}%x
                                {\normalfont\large\bfseries}}
\renewcommand{\subsection}{\@startsection{subsection}{2}{0mm}%
                                {-1explus -.5ex minus -.2ex}%
                                {0.5ex plus .2ex}%
                                {\normalfont\normalsize\bfseries}}
\renewcommand{\subsubsection}{\@startsection{subsubsection}{3}{0mm}%
                                {-1ex plus -.5ex minus -.2ex}%
                                {1ex plus .2ex}%
                                {\normalfont\small\bfseries}}
\makeatother
\setcounter{secnumdepth}{0}
\setlength{\parindent}{0pt}
\setlength{\parskip}{0pt plus 0.5ex}

% -----------------------------------------------------------------------

\begin{document}

\raggedright
\footnotesize

\begin{center}
    {\Large\textbf{PC1101 Cheatsheet, by randomwish}}\\[0.5em]
    \url{https://github.com/randomwish/schoolNotes}
\end{center}
\begin{multicols}{3}
\setlength{\premulticols}{1pt}
\setlength{\postmulticols}{1pt}
\setlength{\multicolsep}{1pt}
\setlength{\columnsep}{2pt}

\section{1. Intro to light}
\subsection{Preliminaries}
\subsubsection{Electromagnetic field and Maxwell's equations}
Nomenclature
\begin{itemize}
    \item $\vec{E}$: direction along which a charged particle is pushed (polar vector). Perpendicular to direction of propagation.
    \item $\vec{D}$: electric displacement field.
    \item $\vec{H}$: magnetic field.
    \item $\vec{B}$: axis around which a charged particle will rotate (axial vector). Perpendicular to both $\vec{E}$ and direction of propagation.
    \item $\nabla$: derivatives with respect to spatial coordinates.
    \item $\rho$: charge density.
    \item $\vec{j}$: current density.
\end{itemize}

Equations
\begin{itemize}
    \item $\vec{\nabla} \times \vec{E}(\vec{x}, t) + \partial_{t} \vec{B}(\vec{x}, t) = 0$
    \begin{itemize}
        \item $\vec{E}$ circulates around any region where $\vec{B}$ is changing with time.
    \end{itemize}
    \item $\vec{\nabla} \cdot \vec{B}(\vec{x},t) = 0$
    \begin{itemize}
        \item No magnetic monopoles; net outflow of $\vec{B}$ is zero.
    \end{itemize}
    \item $\vec{\nabla} \cdot \vec{D}(\vec{x},t) = \rho(\vec{x},t)$
    \begin{itemize}
        \item The electric displacement field $\vec{D}$ relates to charge density.
    \end{itemize}
    \item $\vec{\nabla} \times \vec{H}(\vec{x},t) - \partial_{t} \vec{D}(\vec{x},t) = \vec{j}(\vec{x},t)$
    \begin{itemize}
        \item Describes the circulation of $\vec{H}$ around a current density $\vec{j}$ and the time derivative of $\vec{D}$.
    \end{itemize}
\end{itemize}

\subsection{E-m field with matter}
\subsubsection{Forces on charged particle}
In addition to factoring in Newton's equation with electric and Lorentz forces, include the Abraham-Lorentz force, which accounts for the energy radiated by an accelerated charge.

\subsubsection{Interaction with a molecule}
An electric dipole arises when positive and negative charges move in opposite directions, causing the centers of mass of both clouds to not overlap.

When an electric field oscillates, the dipole oscillates at the same frequency, emitting a wave at that frequency and scattering in all directions. Electrons are stable only in specific energy states. The energy difference between two states, $\Delta E_{jk} = E_j - E_k$, corresponds to a frequency via the relation
\[
\Delta E_{jk} = h\nu_{jk} - \hbar \omega_{jk},
\]
where $h$ is Planck's constant.

\subsubsection{Interaction with solids}
In dielectric materials, such as glass (where electrons are bound to their atoms), light scattered by each atom remains in phase with the light scattered by other atoms. As a result, light passing through the material continues to propagate in a coherent direction.

\section{Frequency}
The dispersion relation is given by $\lambda v = c$, where $c$ is the speed of light in vacuum ($3 \times 10^8$ m/s). Humans can perceive light only when its wavelength lies within the visible spectrum.

\section{2. Monochromatic Waves}
Monochromatic: single color $\rightarrow$ wave has a single frequency.
\begin{itemize}
    \item Frequency $v$: units in Hz.
    \item Phase $\varphi$: only matters if there are two or more waves (relative phase).
    \item Pulsation $\omega = 2 \pi v$: units in rad/s.
    \item Time dependence: $\cos(2\pi v t + \varphi) \equiv \cos(\omega t + \varphi)$.
    \item Conversion: $\sin(\theta) = \cos(\theta - \frac{\pi}{2})$.
\end{itemize}

\subsection{Space Dependence}
\subsubsection{1D Monochromatic Waves}
We represent space as $x - ct$, where $t$ represents time coordinates.

Space and time dependence of a 1D monochromatic wave:
\[
\cos(kx - \omega t + \varphi),
\]
where $k$ is the wave number:
\[
k = \frac{\omega}{c} = \frac{2\pi}{\lambda}.
\]
The minus sign indicates that the wave moves to the right (positive direction).

\subsubsection{3D Monochromatic Waves}
\begin{itemize}
    \item \textbf{Plane wave}: $\cos(\vec{k}\cdot\vec{x} - \omega t + \varphi)$.
    \begin{itemize}
        \item Wavefronts are planes; propagates eternally from $-\infty$ to $\infty$ in the direction $\hat{e_k}$.
        \item The space vibrates in unison in the perpendicular direction to $\hat{e_k}$.
    \end{itemize}
    \item \textbf{Spherical wave}: 
    \[
    \frac{1}{r(\vec{x})} \cos(kr(\vec{x}) - \omega t + \varphi),
    \]
    where $k$ is the wave number and $r(\vec{x})$ is the distance of $\vec{x}$ from the center $\vec{x_0}$ where the wave originates.
    \[
    r(\vec{x}) = \sqrt{(x - x_0)^2 + (y - y_0)^2 + (z - z_0)^2}
    \]
    \item wave fronts are surfaces of constant $r$, or spheres centered at $\vec{x_0}$
\end{itemize}

\subsection{Electric Field Representation}

To represent light as an electromagnetic wave, we need to add:

1. Amplitude $E_0$: Gives the wave units of electric field (V/m) and describes how large the field is.

2. Polarization: Direction of oscillation of the electric field vector.

For a monochromatic plane wave propagating along $\hat{e_z}$:

\subsubsection{Linear Polarization}

\[
\vec{E}(\vec{x},t) = E_0 \hat{e}_\theta \cos(kz - \omega t)
\]

Where $\hat{e}_\theta \equiv \cos\theta\hat{e}_x + \sin\theta\hat{e}_y$

\subsubsection{Circular Polarization} 

\[
\vec{E}(\vec{x},t) = E_0[\cos(kz - \omega t)\hat{e}_\theta \pm \sin(kz - \omega t)\hat{e}_{\theta+\pi/2}]
\]

+ sign: right-handed polarization
- sign: left-handed polarization

\subsubsection{Elliptical Polarization}

Most general form:

\[
\vec{E}(\vec{x},t) = E_0[\cos(kz - \omega t)\hat{e}_\theta + \cos(kz - \omega t + \phi)\hat{e}_{\theta+\pi/2}]
\]

\subsection{Intensity}

Intensity at position $\vec{x}$ and time $t$:

\[
I(\vec{x},t) = \vec{E}(\vec{x},t) \cdot \vec{E}(\vec{x},t) \equiv ||\vec{E}(\vec{x},t)||^2
\]

Energy density in vacuum:

\[
u(\vec{x},t) = \frac{1}{2}\epsilon_0(||\vec{E}(\vec{x},t)||^2 + c^2||\vec{B}(\vec{x},t)||^2) \equiv \epsilon_0 I(\vec{x},t)
\]

\subsection{Complex Notation}

Physicists often use complex numbers for calculations:

\[
e^{i\theta} = \cos\theta + i\sin\theta
\]

Advantages:
- Adding phase is multiplication: $e^{i(kx-\omega t)} \rightarrow e^{i\phi}e^{i(kx-\omega t)}$
- Simplifies many calculations

Note: Take real part before computing intensities, since $|e^{i\theta}|^2 = 1$

\subsection{Superposition}

\subsubsection{Linear Superposition}

For two sources producing fields $\vec{E}_1(\vec{x},t)$ and $\vec{E}_2(\vec{x},t)$:

\[
\vec{E}(\vec{x},t) = \vec{E}_1(\vec{x},t) + \vec{E}_2(\vec{x},t)
\]

This is valid in normal media far from saturation.

\subsubsection{Huygens' Principle}
Huygens' principle states that every point on a wavefront acts as a source of secondary spherical wavelets. The new wavefront after a time t is the envelope of these secondary wavelets.
Key points:

Each point on a wavefront is a source of secondary waves
Secondary waves propagate in all directions with the same speed as the primary wave
The new wavefront is the tangent surface to all secondary wavelets


\subsubsection{Diffraction by an Aperture}
Diffraction occurs when waves encounter an obstacle or opening. It's the apparent bending of waves around small obstacles or spreading out past small openings.
\subsubsection{Fraunhofer Diffraction (2D)}
Consider a monochromatic plane wave impinging on a slit of width 2a along the x-direction.
Setup:

Wave propagates in z-direction
Slit width: 2a
Observation point: $\vec{x} = (R\sin\theta, R\cos\theta)$

Using Huygens' principle, we sum the contributions from all points in the slit:
\[
A(\vec{x},t) = \alpha \int_{-a}^{a} dx_0 \frac{1}{((\vec{x}-\vec{x_0})^2)^{1/4}} \cos(k\sqrt{(\vec{x}-\vec{x_0})^2} - \omega t)
]\]
Where $\alpha$ is the infinitesimal amplitude of each Huygens wave.
\subsubsection{Far-field Approximation}
In the far-field (Fraunhofer) limit where $R \gg a$:
\[
A(\vec{x},t) \approx \frac{\alpha}{\sqrt{R}} \int_{-a}^{a} dx_0 \cos(kR - kx_0\sin\theta - \omega t)
\]
Solving this integral leads to:
\[
A(\vec{x},t) = \frac{2a\alpha}{\sqrt{R}} \cos(kR - \omega t) \text{sinc}(ka\sin\theta)
\]
Where $\text{sinc}(u) = \frac{\sin(u)}{u}$
\subsubsection{Interpretation}

For $ka \ll 1$ (slit much smaller than wavelength):

$\text{sinc}(ka\sin\theta) \approx 1$ for all $\theta$
Behaves like a point source


For $ka \gg 1$ (slit much larger than wavelength):

Intensity is significant only when $ka\sin\theta \ll \pi$
Light propagates mostly in the forward direction



\subsection{Extension to 3D}
For a circular aperture in 3D, the diffraction pattern is described by a Bessel function:
\[
A(\vec{x},t) \propto J_1(ka\sin\theta)
\]
Where $J_1$ is the Bessel function of the first kind of order 1.
\subsubsection{Resolution and Rayleigh Criterion}
The Rayleigh criterion for resolution states that two point sources are just resolvable when the central maximum of one diffraction pattern coincides with the first minimum of the other.
For a circular aperture:
\[
\theta_{min} = 1.22 \frac{\lambda}{D}
\]
Where $D$ is the diameter of the aperture.
This criterion sets the fundamental limit for the angular resolution of optical instruments like telescopes and microscopes.
\end{multicols}
\end{document}
